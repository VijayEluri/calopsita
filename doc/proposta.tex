\documentclass[titlepage]{article}
\usepackage{verbatim}
\usepackage[T1]{fontenc}
\usepackage[utf8]{inputenc}
\usepackage[brazil]{babel}
\title{Proposta de Monografia}
\author{}

\begin{document}

\maketitle

\begin{description} 
\item[Alunos:]Cauê Haucke Porta Guerra\\Cecilia Fernandes\\Lucas Cavalcanti dos Santos
\item[Supervisor: ] Prof. Dr. Alfredo Goldman
\end{description}

\section{Tema}
Implementação de um sistema de gerenciamento de projetos para metodologias ágeis leve e flexível, que seja capaz de lidar com diversas metodologias e métricas.


\section{Resumo}
Notando que não há uma solução única que consiga lidar com diversas metodologias ágeis para diferentes projetos, decidimos procurar o que falta nas ferramentas de gerenciamento de projetos ágeis e criar uma ferramenta flexível o bastante para lidar com a rápida evolução das metodologias e suas métricas.

Desenvolveremos essa ferramenta utilizando Java e em sistema de co-orientação, contando com o incentivo dos profissionais da Caelum, como Paulo Silveira (Mestre em Origami Computacional, IME/USP) e Fabio Kung (Engenheiro da Computação, Poli/USP e criador do Jetty Rails). Também procuraremos estar em contato com mestres e mestrandos da área, tais como Danilo Sato, Mariana Bravo e Hugo Corbucci.

Para isso, além de diariamente usar metodologias ágeis, estamos estudando aquelas que ainda não conhecemos. Durante o desenvolvimento da aplicação, estudaremos os padrões de projeto a serem aplicados e arquitetura de sistemas. Também será importante coletar informações sobre IHC\footnote{Interação Homem Computador} para criar uma interface leve e de fácil utilização.

\begin{enumerate}
	\item{\textbf{Introdução} Breve explicação da estrutura da monografia e visão geral do projeto;}
	\item{\textbf{Motivação e público alvo} Alguns fatos e desapontamentos anteriores que nos levaram a desenvolver uma nova ferramenta de gerenciamento de projetos ágeis. Equipes ágeis distribuidas e com clientes fisicamente distantes são o público alvo;}
	\item{\textbf{Desenvolvimento} Como foi o desenvolvimento (TDD, XP com Scrum, integração contínua, etc), quais clientes ágeis participaram, quem ajudou em arquitetura e quem ajudou com código;}
	\item{\textbf{Código} Padrões de projeto utilizados, arquitetura escolhida, linguagem - e porquês;}
	\item{\textbf{Funcionalidades} Quais funcionalidades já foram implementadas e quais seriam as próximas de acordo com o \textit{backlog} do projeto;}
	\item{\textbf{Integração} Integração fácil do Calopsita com uma ferramenta de integração contínua e com outra de comunicação com clientes;}
	\item{\textbf{Visão dos clientes} Pontos positivos e negativos da nossa ferramenta em comparação às utilizadas no mercado segundo os clientes da aplicação;}
	\item{\textbf{Conclusão} Resultados obtidos e satisfação com o projeto. Caminho para continuar a crescer e perspectivas futuras.}
\end{enumerate}

\section{Objetivos}
Nosso objetivo primário é desenvolver uma aplicação leve e de qualidade para gerenciamento de projetos em metodologias ágeis. Sem restringir o software a uma determinada metodologia ou às métricas mais comuns, obter uma aplicação flexível e facilmente estensível.

Pretendemos desenvolver, utilizando práticas ágeis, um gerenciador de projetos que atenda tanto necessidades de equipes comerciais distribuidas, quanto às de projetos ágeis \textit{opensource}, dada a recente adaptação de metodologias como Scrum para esse nicho de \textit{softwares}.

Também é objetivo, fornecer ferramentas para facilitar e melhorar a comunicação entre a equipe de desenvolvimento e o cliente, caso esse não possa estar fisicamente próximo - o que acontece com frequência em equipes distribuídas.

\section{Atividades já realizadas}
Agora em janeiro, já foi discutida a participação da Caelum nesse projeto e acordado que teremos aproximadamente oito horas semanais de trabalho dedicadas a isso, além das horas que dedicaremos fora do horário de trabalho.

O nome do projeto já foi criado (Calopsita), bem como seu repositório no GitHub, que pode ser encontrado no endereço \href{git://github.com/caueguerra/calopsita.git}. Além disso, o gerenciador será desenvolvido em Java para web.

Planejamos começar a programação já nos primeiros dias de fevereiro.

\section{Cronograma}
\begin{tabular}{|l|c|c|c|c|c|c|c|c|c|c|c|c|}
  \hline
  & Jan & Fev & Mar & Abr & Mai & Jun & Jul & Ago & Set & Out & Nov & Dez \\ \hline
  Desenvolvimento	&  & X& X& X& X& X& X&  &  &  &  & \\ \hline
  Monografia 		&  &  &  &  &  &  & X& X& X& X&  & \\ \hline
  Apresentação 		&  &  &  &  &  &  &  &  & X& X& X& \\ \hline
\end{tabular}

\section{Estrutura esperada da parte técnica da monografia}

A monografia terá estrutura similar à seguinte:

\begin{enumerate}
	\item{\textbf{Introdução} Breve explicação da estrutura da monografia e visão geral do projeto;}
	\item{\textbf{Motivação e público alvo} Alguns fatos e desapontamentos anteriores que nos levaram a desenvolver uma nova ferramenta de gerenciamento de projetos ágeis. Equipes ágeis distribuidas e com clientes fisicamente distantes são o público alvo;}
	\item{\textbf{Desenvolvimento} Como foi o desenvolvimento (TDD, XP com Scrum, integração contínua, etc), quais clientes ágeis participaram, quem ajudou em arquitetura e quem ajudou com código;}
	\item{\textbf{Código} Padrões de projeto utilizados, arquitetura escolhida, linguagem - e porquês;}
	\item{\textbf{Funcionalidades} Quais funcionalidades já foram implementadas e quais seriam as próximas de acordo com o \textit{backlog} do projeto;}
	\item{\textbf{Integração} Integração fácil do Calopsita com uma ferramenta de integração contínua e com outra de comunicação com clientes;}
	\item{\textbf{Visão dos clientes} Pontos positivos e negativos da nossa ferramenta em comparação às utilizadas no mercado segundo os clientes da aplicação;}
	\item{\textbf{Conclusão} Resultados obtidos e satisfação com o projeto. Caminho para continuar a crescer e perspectivas futuras;}
	\item{\textbf{Bibliografia} Livros, websites e publicações acadêmicas onde procuramos informações na construção do Calopsita.}
\end{enumerate}

\end{document}
